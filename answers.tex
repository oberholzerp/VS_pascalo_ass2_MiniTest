\documentclass[10pt,a4paper]{article}
\usepackage[latin1]{inputenc}
\usepackage{amsmath}
\usepackage{amsfonts}
\usepackage{amssymb}
\usepackage{makeidx}
\usepackage{graphicx}
\usepackage[left=2.50cm, right=2.50cm, top=2.50cm, bottom=2.50cm]{geometry}

\title{Answers MiniTest}
\author{Joel Busch, Jakob Meier, Pascal Oberholzer}

\begin{document}
	\maketitle
	
	\section*{1. HTTP Protocol Version 1.1}
		\subsection*{a)}
			Request-Line = Method SP Request-URI SP HTTP-Version CRLF \\
			SP: separating character \\
			CRLF: new line in character encoding \\
		
		\subsection*{b)}
			entered in the URL bar: 192.168.1.1:8080 \\
			request line : \\
			\hspace*{5mm}GET / HTTP/1.1 $\backslash r \backslash n$\\
			\hspace*{5mm}Host: 192.168.1.1:8080 $\backslash r \backslash n$\\
			\hspace*{5mm}Connection: close $\backslash r \backslash n \backslash r \backslash n$\\
		
		\subsection*{c)}
			self-descriptive messages: using the HTTP methods and URI\\
			cachability: by using the Cache-Control header fields\\
		
	
	\section*{2. Network I/O}
		\subsection*{a)}
			Sockets: A Socket is bound to a remote host and a port. The connection to the host is created in the background.\\
			Server-Sockets: A Server-Socket is bound to an address of a local interface and a port. Then .accept is called and the call to the accept blocks until a connection to a client is established. Once a connection is established the call to accept returns a socket which represents the connection to a client.\\
		
		\subsection*{b)}
			When we call certain methods on a Stream it might not be able to handle them immediately because it has to wait for an event (I/O) to occur. Therefore it can block (i.e. paused execution) for indefinite time.\\
			InputStream: read\\
			OutputStream: flush, write (in case the buffer is full)\\
		
		
	\section*{3. Representational State Transfer}
		\subsection*{a)}
			false: REST is no protocol
		\subsection*{b)}
			false: All stateless means, the server doesn't store client-context.
		\subsection*{c)}
			Even though they can correspond in practice it's not that strict. Therefore as we understand the statement it is to strong.
		\subsection*{d)}
			It's wrong in the sense that JSON is the only data representation for REST. but it's right in the sense that it can be JSON.
		
	
	\section*{4. WS-* services}
		\subsection*{a)}
			The wsdl file holds the information. It can be retrieved by opening the following URL: \\
			http://vslab.inf.ethz.ch:8080/SunSPOTWebServices/SunSPOTWebservice?wsdl \\
			
			
		\subsection*{b)}
			the type definitions are in the schema location file \\
			$<$xs:element name="getSpot" type="tns:getSpot"$/>$ \\
			$<$xs:element name="getSpotResponse" type="tns:getSpotResponse"$/>$\\
			
		\subsection*{c)}
			The transport protocol is defined in the binding-block of the WSDL file \\
			The soap:address would be in a mailto URL scheme. (e.g. mailto:SunSpotWebservices@vslab.inf.ethz.ch)\\
			
		
	\section*{5. Android Emulator Networking}
		\subsection*{a)}
			Internally it's 10.0.2.15.\\
			Externally it's the IP address of the development machine. \\
			It is always the same, because all Internet traffic is routed through the development machine (by means of network address translation).\\ 
			
		\subsection*{b)}
			It refers to itself (localhost).\\
			
		\subsection*{c)}
			10.0.2.2\\
			
		\subsection*{d)}
			\begin{enumerate}
				\item Connect by telnet to port 5554 on the development machine.
				\item Authenticate using the auth command with the auth token given at the path provided by the shell.
				\item Execute redir add tcp:[port on the emulated device]:[port on the development machine] .
			\end{enumerate}
			Now it's available under the chosen ports on the development machine.
	
\end{document}